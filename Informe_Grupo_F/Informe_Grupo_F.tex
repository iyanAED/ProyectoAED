%  LaTeX support: latex@mdpi.com
%  For support, please attach all files needed for compiling as well as the log file, and specify your operating system, LaTeX version, and LaTeX editor.

%=================================================================
% pandoc conditionals added to preserve backwards compatibility with previous versions of rticles

\documentclass[,,,moreauthors,pdftex]{Definitions/mdpi}


%% Some pieces required from the pandoc template
\setlist[itemize]{leftmargin=*,labelsep=5.8mm}
\setlist[enumerate]{leftmargin=*,labelsep=4.9mm}


%--------------------
% Class Options:
%--------------------
%----------
% journal
%----------
% Choose between the following MDPI journals:
% acoustics, actuators, addictions, admsci, adolescents, aerobiology, aerospace, agriculture, agriengineering, agrochemicals, agronomy, ai, air, algorithms, allergies, alloys, analytica, analytics, anatomia, animals, antibiotics, antibodies, antioxidants, applbiosci, appliedchem, appliedmath, applmech, applmicrobiol, applnano, applsci, aquacj, architecture, arm, arthropoda, arts, asc, asi, astronomy, atmosphere, atoms, audiolres, automation, axioms, bacteria, batteries, bdcc, behavsci, beverages, biochem, bioengineering, biologics, biology, biomass, biomechanics, biomed, biomedicines, biomedinformatics, biomimetics, biomolecules, biophysica, biosensors, biotech, birds, bloods, blsf, brainsci, breath, buildings, businesses, cancers, carbon, cardiogenetics, catalysts, cells, ceramics, challenges, chemengineering, chemistry, chemosensors, chemproc, children, chips, cimb, civileng, cleantechnol, climate, clinpract, clockssleep, cmd, coasts, coatings, colloids, colorants, commodities, compounds, computation, computers, condensedmatter, conservation, constrmater, cosmetics, covid, crops, cryptography, crystals, csmf, ctn, curroncol, cyber, dairy, data, ddc, dentistry, dermato, dermatopathology, designs, devices, diabetology, diagnostics, dietetics, digital, disabilities, diseases, diversity, dna, drones, dynamics, earth, ebj, ecologies, econometrics, economies, education, ejihpe, electricity, electrochem, electronicmat, electronics, encyclopedia, endocrines, energies, eng, engproc, entomology, entropy, environments, environsciproc, epidemiologia, epigenomes, est, fermentation, fibers, fintech, fire, fishes, fluids, foods, forecasting, forensicsci, forests, foundations, fractalfract, fuels, future, futureinternet, futurepharmacol, futurephys, futuretransp, galaxies, games, gases, gastroent, gastrointestdisord, gels, genealogy, genes, geographies, geohazards, geomatics, geosciences, geotechnics, geriatrics, grasses, gucdd, hazardousmatters, healthcare, hearts, hemato, hematolrep, heritage, higheredu, highthroughput, histories, horticulturae, hospitals, humanities, humans, hydrobiology, hydrogen, hydrology, hygiene, idr, ijerph, ijfs, ijgi, ijms, ijns, ijpb, ijtm, ijtpp, ime, immuno, informatics, information, infrastructures, inorganics, insects, instruments, inventions, iot, j, jal, jcdd, jcm, jcp, jcs, jcto, jdb, jeta, jfb, jfmk, jimaging, jintelligence, jlpea, jmmp, jmp, jmse, jne, jnt, jof, joitmc, jor, journalmedia, jox, jpm, jrfm, jsan, jtaer, jvd, jzbg, kidneydial, kinasesphosphatases, knowledge, land, languages, laws, life, liquids, literature, livers, logics, logistics, lubricants, lymphatics, machines, macromol, magnetism, magnetochemistry, make, marinedrugs, materials, materproc, mathematics, mca, measurements, medicina, medicines, medsci, membranes, merits, metabolites, metals, meteorology, methane, metrology, micro, microarrays, microbiolres, micromachines, microorganisms, microplastics, minerals, mining, modelling, molbank, molecules, mps, msf, mti, muscles, nanoenergyadv, nanomanufacturing,\gdef\@continuouspages{yes}} nanomaterials, ncrna, ndt, network, neuroglia, neurolint, neurosci, nitrogen, notspecified, %%nri, nursrep, nutraceuticals, nutrients, obesities, oceans, ohbm, onco, %oncopathology, optics, oral, organics, organoids, osteology, oxygen, parasites, parasitologia, particles, pathogens, pathophysiology, pediatrrep, pharmaceuticals, pharmaceutics, pharmacoepidemiology,\gdef\@ISSN{2813-0618}\gdef\@continuous pharmacy, philosophies, photochem, photonics, phycology, physchem, physics, physiologia, plants, plasma, platforms, pollutants, polymers, polysaccharides, poultry, powders, preprints, proceedings, processes, prosthesis, proteomes, psf, psych, psychiatryint, psychoactives, publications, quantumrep, quaternary, qubs, radiation, reactions, receptors, recycling, regeneration, religions, remotesensing, reports, reprodmed, resources, rheumato, risks, robotics, ruminants, safety, sci, scipharm, sclerosis, seeds, sensors, separations, sexes, signals, sinusitis, skins, smartcities, sna, societies, socsci, software, soilsystems, solar, solids, spectroscj, sports, standards, stats, std, stresses, surfaces, surgeries, suschem, sustainability, symmetry, synbio, systems, targets, taxonomy, technologies, telecom, test, textiles, thalassrep, thermo, tomography, tourismhosp, toxics, toxins, transplantology, transportation, traumacare, traumas, tropicalmed, universe, urbansci, uro, vaccines, vehicles, venereology, vetsci, vibration, virtualworlds, viruses, vision, waste, water, wem, wevj, wind, women, world, youth, zoonoticdis 
% For posting an early version of this manuscript as a preprint, you may use "preprints" as the journal. Changing "submit" to "accept" before posting will remove line numbers.

%---------
% article
%---------
% The default type of manuscript is "article", but can be replaced by: 
% abstract, addendum, article, book, bookreview, briefreport, casereport, comment, commentary, communication, conferenceproceedings, correction, conferencereport, entry, expressionofconcern, extendedabstract, datadescriptor, editorial, essay, erratum, hypothesis, interestingimage, obituary, opinion, projectreport, reply, retraction, review, perspective, protocol, shortnote, studyprotocol, systematicreview, supfile, technicalnote, viewpoint, guidelines, registeredreport, tutorial
% supfile = supplementary materials

%----------
% submit
%----------
% The class option "submit" will be changed to "accept" by the Editorial Office when the paper is accepted. This will only make changes to the frontpage (e.g., the logo of the journal will get visible), the headings, and the copyright information. Also, line numbering will be removed. Journal info and pagination for accepted papers will also be assigned by the Editorial Office.

%------------------
% moreauthors
%------------------
% If there is only one author the class option oneauthor should be used. Otherwise use the class option moreauthors.

%---------
% pdftex
%---------
% The option pdftex is for use with pdfLaTeX. Remove "pdftex" for (1) compiling with LaTeX & dvi2pdf (if eps figures are used) or for (2) compiling with XeLaTeX.

%=================================================================
% MDPI internal commands - do not modify
\firstpage{1} 
\makeatletter 
\setcounter{page}{\@firstpage} 
\makeatother
\pubvolume{1}
\issuenum{1}
\articlenumber{0}
\pubyear{2024}
\copyrightyear{2024}
%\externaleditor{Academic Editor: Firstname Lastname}
\datereceived{ } 
\daterevised{ } % Comment out if no revised date
\dateaccepted{ } 
\datepublished{ } 
%\datecorrected{} % For corrected papers: "Corrected: XXX" date in the original paper.
%\dateretracted{} % For corrected papers: "Retracted: XXX" date in the original paper.
\hreflink{https://doi.org/} % If needed use \linebreak
%\doinum{}
%\pdfoutput=1 % Uncommented for upload to arXiv.org
%\CorrStatement{yes}  % For updates


%=================================================================
% Add packages and commands here. The following packages are loaded in our class file: fontenc, inputenc, calc, indentfirst, fancyhdr, graphicx, epstopdf, lastpage, ifthen, float, amsmath, amssymb, lineno, setspace, enumitem, mathpazo, booktabs, titlesec, etoolbox, tabto, xcolor, colortbl, soul, multirow, microtype, tikz, totcount, changepage, attrib, upgreek, array, tabularx, pbox, ragged2e, tocloft, marginnote, marginfix, enotez, amsthm, natbib, hyperref, cleveref, scrextend, url, geometry, newfloat, caption, draftwatermark, seqsplit
% cleveref: load \crefname definitions after \begin{document}

%=================================================================
% Please use the following mathematics environments: Theorem, Lemma, Corollary, Proposition, Characterization, Property, Problem, Example, ExamplesandDefinitions, Hypothesis, Remark, Definition, Notation, Assumption
%% For proofs, please use the proof environment (the amsthm package is loaded by the MDPI class).

%=================================================================
% Full title of the paper (Capitalized)
\Title{Análisis de la relación entre la actividad turística y el empleo
en el sector servicios en España}

% MDPI internal command: Title for citation in the left column
\TitleCitation{Análisis de la relación entre la actividad turística y el
empleo en el sector servicios en España}

% Author Orchid ID: enter ID or remove command
%\newcommand{\orcidauthorA}{0000-0000-0000-000X} % Add \orcidA{} behind the author's name
%\newcommand{\orcidauthorB}{0000-0000-0000-000X} % Add \orcidB{} behind the author's name


% Authors, for the paper (add full first names)
\Author{Iyán Álvarez, Azahara Martínez, Juan Alcaraz$^{}$, $^{}$}


%\longauthorlist{yes}


% MDPI internal command: Authors, for metadata in PDF
\AuthorNames{Iyán Álvarez, Azahara Martínez, Juan Alcaraz, }

% MDPI internal command: Authors, for citation in the left column

% Affiliations / Addresses (Add [1] after \address if there is only one affiliation.)
\address{%
}

% Contact information of the corresponding author
\corres{Correspondence: }

% Current address and/or shared authorship








% The commands \thirdnote{} till \eighthnote{} are available for further notes

% Simple summary

%\conference{} % An extended version of a conference paper

% Abstract (Do not insert blank lines, i.e. \\)
\abstract{El presente informe analiza la evolución del turismo en España
y su relación con la ocupación en el sector servicios, a partir de datos
oficiales del INE. Se integraron dos bases de datos ---una sobre
viajeros y pernoctaciones, y otra sobre empleo por sectores--- para
realizar un análisis exploratorio, univariante y bivariante a nivel de
comunidad autónoma. Los resultados muestran una clara tendencia
creciente y estacional en la actividad turística, así como una relación
positiva y estadísticamente significativa entre el turismo y el empleo
en la mayoría de las comunidades autónomas. Sin embargo, las
elasticidades obtenidas son bajas, lo que indica que, aunque ambas
variables evolucionan conjuntamente, el empleo responde de forma
inelástica a las variaciones del turismo. En conjunto, los resultados
sugieren que el turismo constituye un motor relevante del empleo en los
servicios, aunque su efecto es moderado y depende de la estructura
económica de cada territorio.}


% Keywords

% The fields PACS, MSC, and JEL may be left empty or commented out if not applicable
%\PACS{J0101}
%\MSC{}
%\JEL{}

%%%%%%%%%%%%%%%%%%%%%%%%%%%%%%%%%%%%%%%%%%
% Only for the journal Diversity
%\LSID{\url{http://}}

%%%%%%%%%%%%%%%%%%%%%%%%%%%%%%%%%%%%%%%%%%
% Only for the journal Applied Sciences

%%%%%%%%%%%%%%%%%%%%%%%%%%%%%%%%%%%%%%%%%%

%%%%%%%%%%%%%%%%%%%%%%%%%%%%%%%%%%%%%%%%%%
% Only for the journal Data



%%%%%%%%%%%%%%%%%%%%%%%%%%%%%%%%%%%%%%%%%%
% Only for the journal Toxins


%%%%%%%%%%%%%%%%%%%%%%%%%%%%%%%%%%%%%%%%%%
% Only for the journal Encyclopedia


%%%%%%%%%%%%%%%%%%%%%%%%%%%%%%%%%%%%%%%%%%
% Only for the journal Advances in Respiratory Medicine
%\addhighlights{yes}
%\renewcommand{\addhighlights}{%

%\noindent This is an obligatory section in “Advances in Respiratory Medicine”, whose goal is to increase the discoverability and readability of the article via search engines and other scholars. Highlights should not be a copy of the abstract, but a simple text allowing the reader to quickly and simplified find out what the article is about and what can be cited from it. Each of these parts should be devoted up to 2~bullet points.\vspace{3pt}\\
%\textbf{What are the main findings?}
% \begin{itemize}[labelsep=2.5mm,topsep=-3pt]
% \item First bullet.
% \item Second bullet.
% \end{itemize}\vspace{3pt}
%\textbf{What is the implication of the main finding?}
% \begin{itemize}[labelsep=2.5mm,topsep=-3pt]
% \item First bullet.
% \item Second bullet.
% \end{itemize}
%}


%%%%%%%%%%%%%%%%%%%%%%%%%%%%%%%%%%%%%%%%%%

% Pandoc syntax highlighting
\usepackage{color}
\usepackage{fancyvrb}
\newcommand{\VerbBar}{|}
\newcommand{\VERB}{\Verb[commandchars=\\\{\}]}
\DefineVerbatimEnvironment{Highlighting}{Verbatim}{commandchars=\\\{\}}
% Add ',fontsize=\small' for more characters per line
\usepackage{framed}
\definecolor{shadecolor}{RGB}{248,248,248}
\newenvironment{Shaded}{\begin{snugshade}}{\end{snugshade}}
\newcommand{\AlertTok}[1]{\textcolor[rgb]{0.94,0.16,0.16}{#1}}
\newcommand{\AnnotationTok}[1]{\textcolor[rgb]{0.56,0.35,0.01}{\textbf{\textit{#1}}}}
\newcommand{\AttributeTok}[1]{\textcolor[rgb]{0.13,0.29,0.53}{#1}}
\newcommand{\BaseNTok}[1]{\textcolor[rgb]{0.00,0.00,0.81}{#1}}
\newcommand{\BuiltInTok}[1]{#1}
\newcommand{\CharTok}[1]{\textcolor[rgb]{0.31,0.60,0.02}{#1}}
\newcommand{\CommentTok}[1]{\textcolor[rgb]{0.56,0.35,0.01}{\textit{#1}}}
\newcommand{\CommentVarTok}[1]{\textcolor[rgb]{0.56,0.35,0.01}{\textbf{\textit{#1}}}}
\newcommand{\ConstantTok}[1]{\textcolor[rgb]{0.56,0.35,0.01}{#1}}
\newcommand{\ControlFlowTok}[1]{\textcolor[rgb]{0.13,0.29,0.53}{\textbf{#1}}}
\newcommand{\DataTypeTok}[1]{\textcolor[rgb]{0.13,0.29,0.53}{#1}}
\newcommand{\DecValTok}[1]{\textcolor[rgb]{0.00,0.00,0.81}{#1}}
\newcommand{\DocumentationTok}[1]{\textcolor[rgb]{0.56,0.35,0.01}{\textbf{\textit{#1}}}}
\newcommand{\ErrorTok}[1]{\textcolor[rgb]{0.64,0.00,0.00}{\textbf{#1}}}
\newcommand{\ExtensionTok}[1]{#1}
\newcommand{\FloatTok}[1]{\textcolor[rgb]{0.00,0.00,0.81}{#1}}
\newcommand{\FunctionTok}[1]{\textcolor[rgb]{0.13,0.29,0.53}{\textbf{#1}}}
\newcommand{\ImportTok}[1]{#1}
\newcommand{\InformationTok}[1]{\textcolor[rgb]{0.56,0.35,0.01}{\textbf{\textit{#1}}}}
\newcommand{\KeywordTok}[1]{\textcolor[rgb]{0.13,0.29,0.53}{\textbf{#1}}}
\newcommand{\NormalTok}[1]{#1}
\newcommand{\OperatorTok}[1]{\textcolor[rgb]{0.81,0.36,0.00}{\textbf{#1}}}
\newcommand{\OtherTok}[1]{\textcolor[rgb]{0.56,0.35,0.01}{#1}}
\newcommand{\PreprocessorTok}[1]{\textcolor[rgb]{0.56,0.35,0.01}{\textit{#1}}}
\newcommand{\RegionMarkerTok}[1]{#1}
\newcommand{\SpecialCharTok}[1]{\textcolor[rgb]{0.81,0.36,0.00}{\textbf{#1}}}
\newcommand{\SpecialStringTok}[1]{\textcolor[rgb]{0.31,0.60,0.02}{#1}}
\newcommand{\StringTok}[1]{\textcolor[rgb]{0.31,0.60,0.02}{#1}}
\newcommand{\VariableTok}[1]{\textcolor[rgb]{0.00,0.00,0.00}{#1}}
\newcommand{\VerbatimStringTok}[1]{\textcolor[rgb]{0.31,0.60,0.02}{#1}}
\newcommand{\WarningTok}[1]{\textcolor[rgb]{0.56,0.35,0.01}{\textbf{\textit{#1}}}}

% tightlist command for lists without linebreak
\providecommand{\tightlist}{%
  \setlength{\itemsep}{0pt}\setlength{\parskip}{0pt}}


% Add imagehandling



\usepackage{longtable}

\begin{document}



%%%%%%%%%%%%%%%%%%%%%%%%%%%%%%%%%%%%%%%%%%

\section{INTRODUCCIÓN}\label{introducciuxf3n}

\subsection{CONTEXTO GENERAL}\label{contexto-general}

El turismo constituye uno de los principales motores económicos de
España, aportando una parte significativa al Producto Interior Bruto y
al empleo nacional. La actividad turística no solo influye en la
generación de riqueza, sino también en el dinamismo de otros sectores,
como el transporte, la hostelería o el comercio.

En los últimos años, la evolución del turismo español ha experimentado
variaciones notables, especialmente durante el periodo marcado por la
pandemia de la COVID-19, que provocó una caída sin precedentes en los
flujos de viajeros y en los niveles de ocupación del sector servicios. A
partir de 2021, el proceso de recuperación ha sido desigual entre las
comunidades autónomas, lo que hace necesario un análisis detallado que
permita comprender las tendencias y relaciones entre las distintas
variables implicadas.

En este contexto, el presente trabajo realiza un análisis exploratorio a
partir de datos oficiales del \textbf{Instituto Nacional de Estadística
(INE)}, con el objetivo de examinar la evolución del turismo y su
relación con la ocupación en el sector servicios. Este enfoque permite
no solo identificar patrones y correlaciones, sino también detectar
posibles valores atípicos derivados de eventos excepcionales, como la
crisis sanitaria, que pueden distorsionar la interpretación de los
resultados.

\subsection{DESCRIPCIÓN DE LOS DATOS}\label{descripciuxf3n-de-los-datos}

Para el desarrollo del presente análisis se han empleado dos bases de
datos principales.

La primera recoge información sobre el turismo en España, desglosada por
comunidades autónomas y provincias, e incluye variables como el número
de viajeros y las pernoctaciones registradas a lo largo de varios años.

La segunda base de datos corresponde al sector servicios, concretamente
a la ocupación laboral dentro del mismo, y se analiza con el propósito
de explorar su relación con la evolución del turismo.

\subsection{CUESTIONES A TRATAR}\label{cuestiones-a-tratar}

En este estudio se abordan \textbf{tres} cuestiones principales:

\begin{enumerate}
\def\labelenumi{\arabic{enumi}.}
\item
  ¿Cómo tratar los valores atípicos (outliers) asociados al periodo de
  la pandemia de la COVID-19 (2019-2021)?
\item
  ¿Qué evolución ha experimentado el turismo en las distintas
  comunidades autónomas a lo largo del periodo analizado?
\item
  ¿Qué relación existe entre la evolución del turismo y la ocupación en
  el sector servicios?
\end{enumerate}

\section{IMPORTACIÓN}\label{importaciuxf3n}

Para garantizar el uso de datos actualizados, la importación se realiza
directamente desde las URLs oficiales del INE, evitando copias locales.

Antes de la carga definitiva, se comprobó la codificación de los
archivos para evitar errores de lectura, optando por el formato
\textbf{UTF-8}.

Finalmente, los datos se importaron y almacenaron en dos data frames
principales: uno para turismo y otro para ocupación.

\section{ANÁLISIS UNIVARIANTE}\label{anuxe1lisis-univariante}

Antes de realizar el análisis univariante, es necesario limpiar y
preparar las bases de datos. A continuación, se procede al tratamiento
individual de cada una: primero la base de turismo y posteriormente la
base de ocupación.

\subsection{TURISMO}\label{turismo}

\subsubsection{Limpieza}\label{limpieza}

Una vez importados los datos, se analizan las características generales
del conjunto correspondiente al turismo en España. Este dataset recoge
el número de viajeros y pernoctaciones por comunidad y provincia a lo
largo del tiempo.

En un inicio, este dataset se forma de 134.820 filas y 8 columnas, cuyo
nombre resulta completamente representativo.

Las variables incluidas son:

\begin{itemize}
\item
  \textbf{Totales.Territoriales}: columna con un único valor (``Total
  Nacional'').
\item
  \textbf{Comunidades.y.Ciudades.Autónomas}: variable categórica que
  representa las comunidades o ciudades autónomas de España.
\item
  \textbf{Provincias}: variable categórica que muestra la provincia
  asociada.
\item
  \textbf{Viajeros.y.pernoctaciones}: variable categórica. Indica el
  tipo de registro (viajeros o pernoctaciones).
\item
  \textbf{Residencia..Nivel.1}: variable categórica redundante que
  contiene el valor ``Total'' en todas sus observaciones.
\item
  \textbf{Residencia..Nivel.2}: variable categórica que indica el origen
  de los viajeros, diferenciando entre residentes en España y residentes
  en el extranjero.
\item
  \textbf{Periodo}: variable tipo fecha que representa el año y mes de
  la observación.
\item
  \textbf{Total}: valor numérico de viajeros o pernoctaciones expresado
  en millones.
\end{itemize}

Las variables Totales.Territoriales, Viajeros.y.pernoctaciones,
Residencia..Nivel.1 y Residencia..Nivel.2, no aportan información
relevante y se eliminan en fases posteriores.

A continuación, se utilizó la función \texttt{glimpse()} del paquete
\texttt{dplyr} para obtener una vista general del dataset, comprobando
que todas las columnas se importaron como tipo carácter y que será
necesario ajustar sus tipos en etapas posteriores.

\begin{Shaded}
\begin{Highlighting}[]
\NormalTok{df\_Turismo }\SpecialCharTok{\%\textgreater{}\%} \FunctionTok{glimpse}\NormalTok{()}
\end{Highlighting}
\end{Shaded}

\begin{verbatim}
## Rows: 134,820
## Columns: 8
## $ Totales.Territoriales            <chr> "Total Nacional", "Total Nacional", "~
## $ Comunidades.y.Ciudades.Autónomas <chr> "", "", "", "", "", "", "", "", "", "~
## $ Provincias                       <chr> "", "", "", "", "", "", "", "", "", "~
## $ Viajeros.y.pernoctaciones        <chr> "Viajero", "Viajero", "Viajero", "Via~
## $ Residencia..Nivel.1              <chr> "Total", "Total", "Total", "Total", "~
## $ Residencia..Nivel.2              <chr> "", "", "", "", "", "", "", "", "", "~
## $ Periodo                          <chr> "2025M09", "2025M08", "2025M07", "202~
## $ Total                            <chr> "12.050.972", "13.828.672", "13.044.7~
\end{verbatim}

Gracias a esto, se observa que todas las columnas se importaron
inicialmente como tipo character. Por ello, en la fase de preparación se
realizará la conversión a tipos de datos adecuados (fechas, factores y
numéricos) para facilitar el análisis estadístico y gráfico.

Para asegurar la correcta manipulación numérica, se eliminaron los
puntos utilizados como separadores de miles en la columna Total.

\begin{Shaded}
\begin{Highlighting}[]
\NormalTok{df\_Turismo }\OtherTok{\textless{}{-}}\NormalTok{ df\_Turismo }\SpecialCharTok{\%\textgreater{}\%} 
  \FunctionTok{mutate}\NormalTok{(}\AttributeTok{Total =} \FunctionTok{str\_replace\_all}\NormalTok{(Total, }\StringTok{"}\SpecialCharTok{\textbackslash{}\textbackslash{}}\StringTok{."}\NormalTok{, }\StringTok{""}\NormalTok{))}
\end{Highlighting}
\end{Shaded}

Posteriormente, se aplicó \texttt{sapply()} junto con \texttt{unique()}
para revisar los valores de cada variable y detectar posibles
inconsistencias.

\begin{Shaded}
\begin{Highlighting}[]
\FunctionTok{sapply}\NormalTok{(df\_Turismo, }\ControlFlowTok{function}\NormalTok{(x) }\FunctionTok{length}\NormalTok{(}\FunctionTok{unique}\NormalTok{(x)))}
\end{Highlighting}
\end{Shaded}

\begin{verbatim}
##            Totales.Territoriales Comunidades.y.Ciudades.Autónomas 
##                                1                               20 
##                       Provincias        Viajeros.y.pernoctaciones 
##                               51                                2 
##              Residencia..Nivel.1              Residencia..Nivel.2 
##                                1                                3 
##                          Periodo                            Total 
##                              321                            95210
\end{verbatim}

Se identificaron varias \textbf{incidencias}: la columna
Totales.Territoriales presentaba un único valor (``Total nacional''), y
Comunidades.y.Ciudades.Autónomas incluía 20 categorías en lugar de 19,
debido a registros vacíos o duplicados.

\begin{Shaded}
\begin{Highlighting}[]
\FunctionTok{print}\NormalTok{(}\FunctionTok{unique}\NormalTok{(df\_Turismo}\SpecialCharTok{$}\NormalTok{Comunidades.y.Ciudades.Autónomas))}
\end{Highlighting}
\end{Shaded}

\begin{verbatim}
##  [1] ""                               "01 Andalucía"                  
##  [3] "02 Aragón"                      "03 Asturias, Principado de"    
##  [5] "04 Balears, Illes"              "05 Canarias"                   
##  [7] "06 Cantabria"                   "07 Castilla y León"            
##  [9] "08 Castilla - La Mancha"        "09 Cataluña"                   
## [11] "10 Comunitat Valenciana"        "11 Extremadura"                
## [13] "12 Galicia"                     "13 Madrid, Comunidad de"       
## [15] "14 Murcia, Región de"           "15 Navarra, Comunidad Foral de"
## [17] "16 País Vasco"                  "17 Rioja, La"                  
## [19] "18 Ceuta"                       "19 Melilla"
\end{verbatim}

Además, se observó que la variable CCAA presentaba registros vacíos o en
blanco, lo que explica las discrepancias detectadas anteriormente. Estos
valores se eliminarán en la fase de limpieza del dataset.

El siguiente paso fue eliminar las filas y columnas redundantes, así
como los valores en blanco, manteniendo únicamente la información
agregada por comunidades autónomas.

\begin{Shaded}
\begin{Highlighting}[]
\NormalTok{df\_Turismo }\OtherTok{\textless{}{-}}\NormalTok{ df\_Turismo }\SpecialCharTok{\%\textgreater{}\%}
  \FunctionTok{select}\NormalTok{(}\SpecialCharTok{{-}}\NormalTok{Totales.Territoriales) }\SpecialCharTok{\%\textgreater{}\%}
  \FunctionTok{filter}\NormalTok{(}\FunctionTok{str\_trim}\NormalTok{(Comunidades.y.Ciudades.Autónomas) }\SpecialCharTok{!=} \StringTok{""}\NormalTok{) }\SpecialCharTok{\%\textgreater{}\%} 
  \FunctionTok{arrange}\NormalTok{(Comunidades.y.Ciudades.Autónomas, Provincias, Periodo)}
\end{Highlighting}
\end{Shaded}

Posteriormente, con el fin de facilitar la interpretación y el análisis
comparativo, se optó por trabajar con los datos agregados a nivel de
comunidades y ciudades autónomas. Este enfoque proporciona una visión
más general del comportamiento del turismo en España, adecuada para su
posterior comparación con la ocupación en el sector servicios.

En consecuencia, se eliminaron las variables no relevantes para este
nivel de análisis (Provincias, Viajeros.y.pernoctaciones,
Residencia..Nivel.1 y Residencia..Nivel.2).

\begin{Shaded}
\begin{Highlighting}[]
\NormalTok{df\_Turismo }\OtherTok{\textless{}{-}}\NormalTok{ df\_Turismo }\SpecialCharTok{\%\textgreater{}\%}
  \FunctionTok{select}\NormalTok{(}\SpecialCharTok{{-}}\FunctionTok{c}\NormalTok{(Provincias,Viajeros.y.pernoctaciones,}
\NormalTok{            Residencia..Nivel}\FloatTok{.1}\NormalTok{,}
\NormalTok{            Residencia..Nivel}\FloatTok{.2}\NormalTok{))}
\end{Highlighting}
\end{Shaded}

Dado que todas las variables se importaron inicialmente como texto, fue
necesario convertir Periodo al formato de fecha. Para ello, se sustituyó
el carácter ``M'' por un guion y se aplicó la función \texttt{ym()} para
reconocer correctamente el año y el mes.

\begin{Shaded}
\begin{Highlighting}[]
\NormalTok{df\_Turismo }\OtherTok{\textless{}{-}}\NormalTok{ df\_Turismo }\SpecialCharTok{\%\textgreater{}\%}
  \FunctionTok{mutate}\NormalTok{(}\AttributeTok{Periodo =} \FunctionTok{str\_replace}\NormalTok{(Periodo, }\StringTok{"M"}\NormalTok{, }\StringTok{"{-}"}\NormalTok{), }
    \AttributeTok{Periodo =} \FunctionTok{ym}\NormalTok{(Periodo))}
\end{Highlighting}
\end{Shaded}

Además, la variable Total se convirtió a tipo numérico para permitir la
realización de cálculos y análisis estadísticos sin errores de formato.

\begin{Shaded}
\begin{Highlighting}[]
\NormalTok{df\_Turismo }\OtherTok{\textless{}{-}}\NormalTok{ df\_Turismo }\SpecialCharTok{\%\textgreater{}\%} 
  \FunctionTok{mutate}\NormalTok{(}\AttributeTok{Total=}\FunctionTok{as.numeric}\NormalTok{(Total))}
\end{Highlighting}
\end{Shaded}

Antes de convertir la variable Comunidades.y.Ciudades.Autónomas a tipo
factor, se limpiaron y estandarizaron sus valores eliminando números,
espacios y caracteres innecesarios.

Finalmente, se renombró como CCAA y se definió como variable categórica
tipo factor.

\begin{Shaded}
\begin{Highlighting}[]
\NormalTok{df\_Turismo }\OtherTok{\textless{}{-}}\NormalTok{ df\_Turismo }\SpecialCharTok{\%\textgreater{}\%} 
  \FunctionTok{rename}\NormalTok{(}\AttributeTok{CCAA =}\NormalTok{ Comunidades.y.Ciudades.Autónomas) }\SpecialCharTok{\%\textgreater{}\%} 
  \FunctionTok{mutate}\NormalTok{(}
    \AttributeTok{CCAA =} \FunctionTok{str\_replace}\NormalTok{(CCAA, }\StringTok{"\^{}[0{-}9]+}\SpecialCharTok{\textbackslash{}\textbackslash{}}\StringTok{s+"}\NormalTok{, }\StringTok{""}\NormalTok{),}
    \AttributeTok{CCAA =} \FunctionTok{str\_replace}\NormalTok{(CCAA,}
                       \StringTok{"\^{}}\SpecialCharTok{\textbackslash{}\textbackslash{}}\StringTok{s*([\^{},]+?)}\SpecialCharTok{\textbackslash{}\textbackslash{}}\StringTok{s*,}\SpecialCharTok{\textbackslash{}\textbackslash{}}\StringTok{s*([\^{},]+?)}\SpecialCharTok{\textbackslash{}\textbackslash{}}\StringTok{s*$"}\NormalTok{, }\StringTok{"}\SpecialCharTok{\textbackslash{}\textbackslash{}}\StringTok{2 }\SpecialCharTok{\textbackslash{}\textbackslash{}}\StringTok{1"}\NormalTok{),}
    \AttributeTok{CCAA =} \FunctionTok{str\_squish}\NormalTok{(CCAA)}
\NormalTok{  ) }\SpecialCharTok{\%\textgreater{}\%}
  \FunctionTok{mutate}\NormalTok{(}\AttributeTok{CCAA =} \FunctionTok{as.factor}\NormalTok{(CCAA))}
\end{Highlighting}
\end{Shaded}

Para continuar con el análisis, se creó un nuevo conjunto de datos
agregado a nivel de comunidad autónoma.

\begin{Shaded}
\begin{Highlighting}[]
\NormalTok{df\_Turismo }\OtherTok{\textless{}{-}}\NormalTok{ df\_Turismo }\SpecialCharTok{\%\textgreater{}\%}
  \FunctionTok{group\_by}\NormalTok{(CCAA, Periodo) }\SpecialCharTok{\%\textgreater{}\%} 
  \FunctionTok{summarise}\NormalTok{(}\AttributeTok{Total =} \FunctionTok{sum}\NormalTok{(Total, }\AttributeTok{na.rm =} \ConstantTok{TRUE}\NormalTok{), }\AttributeTok{.groups =} \StringTok{"drop"}\NormalTok{)}
\end{Highlighting}
\end{Shaded}

Tras la limpieza, se comprobó la ausencia de espacios en blanco y
valores faltantes (NA) para asegurar la integridad del conjunto de
datos. Al no detectarse incidencias, el dataset quedó listo para el
análisis y se guardó en formato RData.

\subsubsection{Detección de outliers}\label{detecciuxf3n-de-outliers}

Para la detección de valores atípicos en las series de turismo se
aplicaron tres métodos estadísticos complementarios:

\begin{itemize}
\tightlist
\item
  Regla 3-sigmas: considera como \emph{outliers} los valores que se
  alejan más de tres desviaciones típicas respecto a la media.\\
\item
  Método de Hampel: utiliza la mediana y la desviación absoluta mediana
  (MAD) para detectar observaciones atípicas, siendo más robusto frente
  a valores extremos.\\
\item
  Método del Boxplot: identifica \emph{outliers} a partir de los límites
  del rango intercuartílico (1,5×IQR).
\end{itemize}

En todos los casos se aplicó una versión estacional del método,
calculando los parámetros (media, desviación, mediana o cuartiles) por
mes y comunidad autónoma, con el objetivo de evitar clasificar como
atípicos los picos o caídas debidos a la estacionalidad turística.

Los gráficos se generaron automáticamente para cada comunidad autónoma,
aunque solo se muestra un ejemplo en el informe. Todos ellos se han
guardado en carpetas independientes:\\
\texttt{graficos\_outliers\_3sigma}, \texttt{graficos\_outliers\_hampel}
y \texttt{graficos\_outliers\_boxplot}.

\begin{Shaded}
\begin{Highlighting}[]
\NormalTok{Sigma3\_est }\OtherTok{\textless{}{-}} \ControlFlowTok{function}\NormalTok{(x, }\AttributeTok{k =} \DecValTok{3}\NormalTok{) \{}
\NormalTok{  media }\OtherTok{\textless{}{-}} \FunctionTok{mean}\NormalTok{(x, }\AttributeTok{na.rm =} \ConstantTok{TRUE}\NormalTok{)}
\NormalTok{  desv }\OtherTok{\textless{}{-}} \FunctionTok{sd}\NormalTok{(x, }\AttributeTok{na.rm =} \ConstantTok{TRUE}\NormalTok{)}
\NormalTok{  out }\OtherTok{\textless{}{-}} \FunctionTok{abs}\NormalTok{(x }\SpecialCharTok{{-}}\NormalTok{ media) }\SpecialCharTok{\textgreater{}}\NormalTok{ (k }\SpecialCharTok{*}\NormalTok{ desv)}
\NormalTok{\}}
\end{Highlighting}
\end{Shaded}

\includegraphics[width=1\linewidth,height=0.5\textheight]{Informe_Grupo_F_files/figure-latex/unnamed-chunk-13-1}

\begin{Shaded}
\begin{Highlighting}[]
\NormalTok{Hampel\_est }\OtherTok{\textless{}{-}} \ControlFlowTok{function}\NormalTok{(x, }\AttributeTok{k =} \DecValTok{3}\NormalTok{) \{}
  \FunctionTok{abs}\NormalTok{(x }\SpecialCharTok{{-}} \FunctionTok{median}\NormalTok{(x)) }\SpecialCharTok{\textgreater{}}\NormalTok{ k }\SpecialCharTok{*} \FunctionTok{mad}\NormalTok{(x)}
\NormalTok{\}}
\end{Highlighting}
\end{Shaded}

\includegraphics[width=1\linewidth,height=0.3\textheight]{Informe_Grupo_F_files/figure-latex/unnamed-chunk-16-1}

\begin{Shaded}
\begin{Highlighting}[]
\NormalTok{Boxplot\_est }\OtherTok{\textless{}{-}} \ControlFlowTok{function}\NormalTok{(x, }\AttributeTok{k =} \FloatTok{1.5}\NormalTok{) \{}
\NormalTok{  Q1 }\OtherTok{\textless{}{-}} \FunctionTok{quantile}\NormalTok{(x, }\FloatTok{0.25}\NormalTok{, }\AttributeTok{na.rm =} \ConstantTok{TRUE}\NormalTok{)}
\NormalTok{  Q3 }\OtherTok{\textless{}{-}} \FunctionTok{quantile}\NormalTok{(x, }\FloatTok{0.75}\NormalTok{, }\AttributeTok{na.rm =} \ConstantTok{TRUE}\NormalTok{)}
\NormalTok{  IQR\_value }\OtherTok{\textless{}{-}}\NormalTok{ Q3 }\SpecialCharTok{{-}}\NormalTok{ Q1}
\NormalTok{  inf }\OtherTok{\textless{}{-}}\NormalTok{ Q1 }\SpecialCharTok{{-}}\NormalTok{ k }\SpecialCharTok{*}\NormalTok{ IQR\_value}
\NormalTok{  sup }\OtherTok{\textless{}{-}}\NormalTok{ Q3 }\SpecialCharTok{+}\NormalTok{ k }\SpecialCharTok{*}\NormalTok{ IQR\_value}
\NormalTok{  (x }\SpecialCharTok{\textless{}}\NormalTok{ inf) }\SpecialCharTok{|}\NormalTok{ (x }\SpecialCharTok{\textgreater{}}\NormalTok{ sup)}
\NormalTok{\}}
\end{Highlighting}
\end{Shaded}

\includegraphics[width=1\linewidth,height=0.3\textheight]{Informe_Grupo_F_files/figure-latex/unnamed-chunk-18-1}

\begin{verbatim}
## # A tibble: 19 x 4
##    CCAA                       n_hampel n_boxplot n_sigma3
##    <fct>                         <int>     <int>    <int>
##  1 Ceuta                            20        21        8
##  2 Illes Balears                    18        17        8
##  3 Castilla - La Mancha             14        16       11
##  4 Castilla y León                  13        14        9
##  5 Comunitat Valenciana             12        13       10
##  6 La Rioja                         12        15       10
##  7 Región de Murcia                 12        13        8
##  8 Cantabria                        11        13        7
##  9 Andalucía                        10        12        9
## 10 Extremadura                      10        11        7
## 11 Aragón                            9         9        6
## 12 Galicia                           9        12        6
## 13 Melilla                           7         4        1
## 14 Principado de Asturias            6         9        3
## 15 Cataluña                          5         6        3
## 16 Comunidad Foral de Navarra        1         1        0
## 17 Canarias                          0         0        0
## 18 Comunidad de Madrid               0         0        0
## 19 País Vasco                        0         0        0
\end{verbatim}

Durante la aplicación de los distintos métodos de detección de valores
atípicos (3-sigma, Hampel y Boxplot) se observó que algunas comunidades
autónomas no presentaban \emph{outliers}, lo cual resultaba poco
coherente con la evolución visible de sus series temporales.

Para contrastar esta situación, se analizaron los estadísticos
descriptivos de los valores mensuales (media, varianza y desviación
típica). Este análisis permitió comprobar que, en algunos casos, como el
de la Comunidad de Madrid, la desviación típica estacional es elevada
debido a un crecimiento sostenido en los últimos años.

Esto implica que, aunque la variabilidad dentro de cada año es baja, la
variabilidad entre estaciones es alta, provocando que el método 3-sigma
no identifique correctamente ciertos valores que visualmente podrían
considerarse atípicos.

Este comportamiento también se repite en otras comunidades de similares
características, donde las medias y la dispersión entre estaciones son
altas.

Dado que los resultados sugieren que los métodos tradicionales no captan
adecuadamente todas las anomalías, y teniendo en cuenta el conocimiento
contextual del fenómeno turístico, se ha decidido considerar como
valores atípicos comunes el periodo comprendido entre marzo de 2020 y
febrero de 2021, correspondiente al impacto más severo de la pandemia de
la COVID-19.

Este enfoque permite homogeneizar el tratamiento de outliers en todas
las comunidades, evitando que diferencias en la dispersión estacional
alteren la comparación entre regiones.

\subsubsection{Análisis de series de
tiempo}\label{anuxe1lisis-de-series-de-tiempo}

Los datos de turismo referentes al periodo de la pandemia son claramente
\emph{outlier}, la pregunta principal a la que se debe hacer frente es
si eliminarlo o si tenerlos en cuenta para posteriores análisis. Esta
pregunta no es sencilla pues para ciertas CCAA, como se ha mencionado,
hay métodos de detección de \emph{outliers} que no los clasifican como
tales.

La idea para justificar qué debemos hacer con estos datos anómalos es la
siguiente, en general, si representamos la serie temporal de una
comunidad autónoma se puede ver que la gran bajada del número de
pernoctaciones a partir del mes de marzo de 2020, lo cual está
directamente relacionado con el inicio del estado de alarma y la
cuarentena nacional.

Tras esto se mantienen unas cifras muy bajas hasta aproximadamente dos
años después (dependiendo de la comunidad), coincidiendo con la
declaración del final de la pandemia por parte de la OMS, los datos de
estancias vuelven prácticamente a los niveles prepandemia, lo cual
sugiere que aunque la pandemia afectó de forma muy negativa al sector de
la hostelería y el turismo, también este sector recuperó de forma
bastante rápida cuando la situación sanitaria lo permitió.

Esto hace pensar que la pandemia simplemente puso una ``pausa'' al
turismo y que tras esta pausa la actividad en este sector continuó
desarrollándose de la misma forma que lo venía haciendo en los últimos
años, aunque si es cierto que el crecimiento se desaceleró ligeramente.
Lo cual sugiere que los datos de la pandemia no aportan ningún tipo de
información relevante que pueda ayudar a predecir o explicar el futuro
del sector del turismo.

Pues no solo no aportan información relevante, si no que además
añadirían ruido a los datos afectando de forma negativa a los posibles
análisis que puedan hacerse.

Para darle respaldo a esta hipótesis nos proponemos hacer un análisis de
la serie de datos de cada CCAA, cosa que es muy natural por la forma en
la que están organizados los datos.

Entonces, se tomarán los datos previos a la pandemia y se utilizarán
para crear un modelo \(ARIMA(p, d, q)(P, D, Q)\), el ajuste de los
parámetros se hace de forma automática con las función
\texttt{auto.arima}. Tras el ajuste se usa la función \texttt{forecast}
que permite utilizar el modelo para predecir una serie de valores en el
futuro, en este caso de predicen los \(12\) meses siguientes.

Los resultados para cada comunidad se guardaron en un archivo llamado
\texttt{AnalisisBox.rsd}. Se trata de una lista, si queremos acceder a
ellos se pueden importar con el comando
\texttt{readRSD("AnálisisBox.rsd")} y asignándolo a una variable.

Se pueden ver las dierentes gráficas de juste para cada comunidad, como
son muchos gráficos se muestra solo uno, pero se puede ver que el resto
muestran resultados similares. Ponemos el caso de \textbf{Andalucía}

\begin{Shaded}
\begin{Highlighting}[]
\NormalTok{AnalisisBox[[}\DecValTok{2}\NormalTok{]]}
\end{Highlighting}
\end{Shaded}

\includegraphics[width=1\linewidth,height=0.3\textheight]{Informe_Grupo_F_files/figure-latex/Gráfica de ajuste-1}
Entonces, tras ver que el modelo aproxima de forma razonable los datos
prepandemia, hay razones para confiar en las predicciones, las cuales se
pueden interpretar como los datos de pernoctaciones que hubieran
sucedido en esos meses siguientes a febrero de \(2020\) si la pandemia
no hubiera azotado.

Lo más interesante ocurre si comparamos esas predicciones con los datos
posteriores a la pandemia. Si bien es cierto que hay algunas comunidades
para las cuales las predicciones no se ajustan en nada (son los casos de
las ciudades autónomas) en el resto de casos se puede ver como la
predicción y el valor postpandemia son iguales solo que con cifras algo
menores menores.

\includegraphics[width=1\linewidth,height=0.3\textheight]{Informe_Grupo_F_files/figure-latex/Gráfica de predicción buena-1}

\includegraphics{Informe_Grupo_F_files/figure-latex/Gráfica de predicción intermedia-1.pdf}

\includegraphics[width=1\linewidth,height=0.3\textheight]{Informe_Grupo_F_files/figure-latex/Gráfica de predicción mala-1}

\subsubsection{Análisis}\label{anuxe1lisis}

Como primer paso del análisis univariante, se calcularon los
estadísticos descriptivos básicos con el fin de obtener una visión
general de las variables del conjunto de datos.

\begin{Shaded}
\begin{Highlighting}[]
\FunctionTok{summary}\NormalTok{(df\_Turismo)}
\end{Highlighting}
\end{Shaded}

\begin{verbatim}
##                    CCAA         Periodo               Total         
##  Andalucía           : 321   Min.   :1999-01-01   Min.   :       0  
##  Aragón              : 321   1st Qu.:2005-09-01   1st Qu.:  818367  
##  Canarias            : 321   Median :2012-05-01   Median : 1978968  
##  Cantabria           : 321   Mean   :2012-05-01   Mean   : 6265921  
##  Castilla - La Mancha: 321   3rd Qu.:2019-01-01   3rd Qu.: 7983666  
##  Castilla y León     : 321   Max.   :2025-09-01   Max.   :54032304  
##  (Other)             :4173
\end{verbatim}

El resumen estadístico muestra que todas las comunidades autónomas
tienen el mismo número de observaciones, con registros trimestrales
entre 1999 y 2025. La variable Total presenta un mínimo de 0
(probablemente asociado a la pandemia), una media de 6,27 millones y un
máximo de 54 millones, reflejando una notable variabilidad entre
regiones y periodos.

Dado que la función \texttt{summary()} ofrece información limitada para
variables categóricas, se amplió el análisis calculando los principales
estadísticos descriptivos de Total por comunidad autónoma. Este enfoque
permite comparar la magnitud y variabilidad del turismo entre regiones,
antes de profundizar en las relaciones en el análisis bivariante.

\begin{Shaded}
\begin{Highlighting}[]
\NormalTok{df\_Turismo }\SpecialCharTok{\%\textgreater{}\%}
  \FunctionTok{group\_by}\NormalTok{(CCAA) }\SpecialCharTok{\%\textgreater{}\%}
  \FunctionTok{summarise}\NormalTok{(}
    \AttributeTok{media =} \FunctionTok{mean}\NormalTok{(Total, }\AttributeTok{na.rm =} \ConstantTok{TRUE}\NormalTok{),}
    \AttributeTok{mediana =} \FunctionTok{median}\NormalTok{(Total, }\AttributeTok{na.rm =} \ConstantTok{TRUE}\NormalTok{),}
    \AttributeTok{sd =} \FunctionTok{sd}\NormalTok{(Total, }\AttributeTok{na.rm =} \ConstantTok{TRUE}\NormalTok{),}
    \AttributeTok{min =} \FunctionTok{min}\NormalTok{(Total, }\AttributeTok{na.rm =} \ConstantTok{TRUE}\NormalTok{),}
    \AttributeTok{max =} \FunctionTok{max}\NormalTok{(Total, }\AttributeTok{na.rm =} \ConstantTok{TRUE}\NormalTok{),}
    \AttributeTok{IQR=}\FunctionTok{IQR}\NormalTok{(Total, }\AttributeTok{na.rm=}\ConstantTok{TRUE}\NormalTok{),}
    \AttributeTok{N =} \FunctionTok{n}\NormalTok{()}
\NormalTok{  ) }\SpecialCharTok{\%\textgreater{}\%}
  \FunctionTok{arrange}\NormalTok{(}\FunctionTok{desc}\NormalTok{(media)) }\SpecialCharTok{\%\textgreater{}\%} 
  \FunctionTok{print}\NormalTok{()}
\end{Highlighting}
\end{Shaded}

\begin{verbatim}
## # A tibble: 19 x 8
##    CCAA                           media mediana     sd   min    max    IQR     N
##    <fct>                          <dbl>   <dbl>  <dbl> <dbl>  <dbl>  <dbl> <int>
##  1 Cataluña                   20162927.  1.82e7 1.11e7     0 4.62e7 1.74e7   321
##  2 Illes Balears              19915553.  1.41e7 1.73e7     0 5.40e7 3.30e7   321
##  3 Canarias                   19889672.  2.02e7 6.40e6     0 3.25e7 9.84e6   321
##  4 Andalucía                  19415478.  1.88e7 8.28e6     0 3.95e7 1.29e7   321
##  5 Comunitat Valenciana       10438034.  1.00e7 4.12e6     0 2.10e7 5.60e6   321
##  6 Comunidad de Madrid         9063390.  9.19e6 2.98e6     0 1.51e7 4.74e6   321
##  7 Castilla y León             3752446.  3.81e6 1.26e6     0 6.91e6 1.75e6   321
##  8 Galicia                     3739708.  3.21e6 1.98e6     0 9.75e6 2.63e6   321
##  9 País Vasco                  2295078.  2.11e6 1.00e6     0 5.41e6 1.33e6   321
## 10 Aragón                      2254042.  2.18e6 7.28e5     0 4.60e6 8.69e5   321
## 11 Castilla - La Mancha        1767775.  1.82e6 4.56e5     0 2.81e6 6.30e5   321
## 12 Principado de Asturias      1506391.  1.33e6 8.70e5     0 4.02e6 1.11e6   321
## 13 Región de Murcia            1270981.  1.27e6 4.45e5     0 2.54e6 6.22e5   321
## 14 Cantabria                   1212464.  1.04e6 7.35e5     0 3.19e6 9.70e5   321
## 15 Extremadura                 1078914.  1.09e6 3.26e5     0 1.89e6 4.72e5   321
## 16 Comunidad Foral de Navarra   766060.  7.27e5 3.13e5     0 1.63e6 4.71e5   321
## 17 La Rioja                     458772.  4.65e5 1.54e5     0 9.40e5 2.12e5   321
## 18 Ceuta                         37138.  3.78e4 9.27e3     0 6.47e4 9.14e3   321
## 19 Melilla                       27683.  2.80e4 8.34e3     0 4.88e4 1.12e4   321
\end{verbatim}

Los resultados muestran que Cataluña presenta la media más alta en el
número de turistas, lo que refleja su consolidación como uno de los
principales destinos turísticos del país. Le siguen Baleares, Canarias y
Andalucía, todas ellas comunidades con una fuerte especialización en el
sector turístico.

En el extremo opuesto, Melilla, Ceuta, La Rioja y Navarra registran las
medias más bajas, lo que podría deberse a su menor capacidad de
atracción turística o a un peso económico menos centrado en este sector.

Para obtener una visión general del comportamiento del turismo en el
conjunto del país, se calcularon los principales estadísticos
descriptivos del Total Nacional, agregando los valores de todas las
comunidades por periodo.

\begin{Shaded}
\begin{Highlighting}[]
\NormalTok{nacional\_mensual }\OtherTok{\textless{}{-}}\NormalTok{ df\_Turismo }\SpecialCharTok{\%\textgreater{}\%}
  \FunctionTok{group\_by}\NormalTok{(Periodo) }\SpecialCharTok{\%\textgreater{}\%}
  \FunctionTok{summarise}\NormalTok{(}\AttributeTok{Total =} \FunctionTok{sum}\NormalTok{(Total, }\AttributeTok{na.rm =} \ConstantTok{TRUE}\NormalTok{), }\AttributeTok{.groups =} \StringTok{"drop"}\NormalTok{)}

\NormalTok{nacional\_mensual }\SpecialCharTok{\%\textgreater{}\%}
  \FunctionTok{summarise}\NormalTok{(}
    \AttributeTok{media =} \FunctionTok{mean}\NormalTok{(Total, }\AttributeTok{na.rm =} \ConstantTok{TRUE}\NormalTok{),}
    \AttributeTok{mediana =} \FunctionTok{median}\NormalTok{(Total, }\AttributeTok{na.rm =} \ConstantTok{TRUE}\NormalTok{),}
    \AttributeTok{sd =} \FunctionTok{sd}\NormalTok{(Total, }\AttributeTok{na.rm =} \ConstantTok{TRUE}\NormalTok{),}
    \AttributeTok{var =} \FunctionTok{var}\NormalTok{(Total, }\AttributeTok{na.rm =} \ConstantTok{TRUE}\NormalTok{),}
    \AttributeTok{min =} \FunctionTok{min}\NormalTok{(Total, }\AttributeTok{na.rm =} \ConstantTok{TRUE}\NormalTok{),}
    \AttributeTok{max =} \FunctionTok{max}\NormalTok{(Total, }\AttributeTok{na.rm =} \ConstantTok{TRUE}\NormalTok{),}
    \AttributeTok{N =} \FunctionTok{n}\NormalTok{()}
\NormalTok{  )}
\end{Highlighting}
\end{Shaded}

\begin{verbatim}
## # A tibble: 1 x 7
##        media   mediana        sd     var   min       max     N
##        <dbl>     <dbl>     <dbl>   <dbl> <dbl>     <dbl> <int>
## 1 119052505. 111782620 52682662. 2.78e15     0 247949935   321
\end{verbatim}

Los resultados agregados a nivel nacional muestran una media de
aproximadamente 119 millones de turistas por periodo y una mediana
cercana a 112 millones, lo que indica una distribución relativamente
equilibrada aunque con cierta asimetría hacia los valores altos.

La desviación típica, de unos 52 millones, junto con una varianza
elevada, refleja una alta variabilidad en el volumen de turismo entre
periodos.

El valor mínimo (0) corresponde al impacto de la pandemia de la
COVID-19, mientras que el máximo (247,9 millones) evidencia los picos
alcanzados en los años de mayor actividad turística.

\begin{center}\rule{0.5\linewidth}{0.5pt}\end{center}

Para ilustrar la evolución del turismo en el conjunto nacional, se
elaboró una serie temporal que permite observar las principales
tendencias y patrones estacionales.

\begin{Shaded}
\begin{Highlighting}[]
\FunctionTok{ggplot}\NormalTok{(nacional\_mensual, }\FunctionTok{aes}\NormalTok{(Periodo, Total)) }\SpecialCharTok{+}
  \FunctionTok{geom\_line}\NormalTok{(}\AttributeTok{linewidth =} \FloatTok{0.8}\NormalTok{) }\SpecialCharTok{+}
  \FunctionTok{scale\_y\_continuous}\NormalTok{(}\AttributeTok{labels =} \FunctionTok{label\_comma}\NormalTok{(}\AttributeTok{big.mark =} \StringTok{"."}\NormalTok{, }\AttributeTok{decimal.mark =} \StringTok{","}\NormalTok{))}\SpecialCharTok{+}
  \FunctionTok{labs}\NormalTok{(}\AttributeTok{title =} \StringTok{"Serie temporal de viajeros por mes"}\NormalTok{,}
       \AttributeTok{x =} \StringTok{"Periodo"}\NormalTok{, }\AttributeTok{y =} \StringTok{"Total"}\NormalTok{) }\SpecialCharTok{+}
  \FunctionTok{theme\_minimal}\NormalTok{()}
\end{Highlighting}
\end{Shaded}

\includegraphics[width=1\linewidth,height=0.3\textheight]{Informe_Grupo_F_files/figure-latex/unnamed-chunk-24-1}

La gráfica muestra una tendencia general al alza con una clara
estacionalidad anual, marcada por picos recurrentes en los meses de
mayor actividad turística. También se identifican valores atípicos
durante el periodo de la pandemia de la COVID-19, que reflejan la brusca
caída del sector en esos años.

A continuación, se representa la serie temporal de cada comunidad y
ciudad autónoma para analizar la evolución del turismo y comparar sus
comportamientos a lo largo del tiempo.

\begin{Shaded}
\begin{Highlighting}[]
\NormalTok{ca\_mensual }\OtherTok{\textless{}{-}}\NormalTok{ df\_Turismo }\SpecialCharTok{\%\textgreater{}\%}
  \FunctionTok{group\_by}\NormalTok{(CCAA, Periodo) }\SpecialCharTok{\%\textgreater{}\%}
  \FunctionTok{summarise}\NormalTok{(}\AttributeTok{Total =} \FunctionTok{sum}\NormalTok{(Total, }\AttributeTok{na.rm =} \ConstantTok{TRUE}\NormalTok{), }\AttributeTok{.groups =} \StringTok{"drop"}\NormalTok{)}

\FunctionTok{ggplot}\NormalTok{(ca\_mensual, }\FunctionTok{aes}\NormalTok{(Periodo, Total)) }\SpecialCharTok{+}
  \FunctionTok{geom\_line}\NormalTok{(}\AttributeTok{linewidth =} \FloatTok{0.6}\NormalTok{) }\SpecialCharTok{+}
  \FunctionTok{facet\_wrap}\NormalTok{(}\SpecialCharTok{\textasciitilde{}}\NormalTok{ CCAA, }\AttributeTok{scales =} \StringTok{"free\_y"}\NormalTok{, }\AttributeTok{ncol =} \DecValTok{3}\NormalTok{) }\SpecialCharTok{+}
  \FunctionTok{scale\_y\_continuous}\NormalTok{(}\AttributeTok{labels =} \FunctionTok{label\_comma}\NormalTok{(}\AttributeTok{big.mark =} \StringTok{"."}\NormalTok{, }\AttributeTok{decimal.mark =} \StringTok{","}\NormalTok{))}\SpecialCharTok{+}
  \FunctionTok{labs}\NormalTok{(}\AttributeTok{title =} \StringTok{"Serie temporal por Comunidad Autónoma"}\NormalTok{,}
       \AttributeTok{x =} \StringTok{"Periodo"}\NormalTok{, }\AttributeTok{y =} \StringTok{"Total"}\NormalTok{) }\SpecialCharTok{+}
  \FunctionTok{theme\_minimal}\NormalTok{()}
\end{Highlighting}
\end{Shaded}

\includegraphics[width=1\linewidth,height=0.3\textheight]{Informe_Grupo_F_files/figure-latex/unnamed-chunk-25-1}

En general, todas las comunidades muestran una marcada estacionalidad,
con picos de actividad durante los meses de verano. A largo plazo,
Madrid, el País Vasco y Canarias presentan una tendencia claramente
creciente, mientras que otras, como Ceuta, mantienen una evolución más
estable.

Asimismo, se observa heterocedasticidad en regiones como Illes Balears,
Asturias, Andalucía o la Comunitat Valenciana, donde la variabilidad
aumenta con el número de turistas, a diferencia de Madrid o Canarias,
donde la dispersión se mantiene más constante.

Por último, todas las comunidades registran una caída abrupta en 2020,
asociada al impacto de la pandemia de la COVID-19. En el siguiente
apartado se aplicará el método 3sigma estacional (por mes y comunidad
autónoma) para identificar de forma formal los valores atípicos sin
confundirlos con la estacionalidad natural de la serie.

\#\#TODO METER LOS MAPAS Y EXPLICARLO Mas bien, meter el del año 2024 y
explicarlo. También estaría bien meter algo más de bivariante

\subsection{OCUPACIÓN}\label{ocupaciuxf3n}

El segundo conjunto de datos corresponde a la ocupación por sectores
económicos en España, obtenido del INE. Contiene el número de personas
ocupadas (en miles) por comunidad o ciudad autónoma, sector y periodo
trimestral.

El dataset está compuesto por 21.300 observaciones (filas) y 5 variables
(columnas), cuyos nombres son claros y representativos, lo que facilita
su comprensión y manipulación posterior.

Las variables incluidas son:

\begin{itemize}
\item
  \textbf{Sexo}. Variable categórica que presenta el valor ``mujer'',
  ``hombre'' y ``ambos sexos''.
\item
  \textbf{Comunidades.y.Ciudades.Autónomas}. Variable categórica que
  incluye las comunidades Españolas y el total nacional.
\item
  \textbf{Sector.económico}. Variable categórica que clasifica la
  actividad económica en agricultura, industria, construcción o
  servicios.
\item
  \textbf{Periodo}. Variable tipo fecha que muestra el año y trimestre.
\item
  \textbf{Total}. Variable numérica que muestra las personas ocupadas
  (en miles).
\end{itemize}

Como primer paso, se utilizó la función \texttt{glimpse()} para revisar
la estructura del dataset. Se comprobó que todas las columnas se
importaron como tipo carácter, por lo que será necesario ajustar los
tipos de datos en los siguientes pasos.

\subsubsection{Limpieza}\label{limpieza-1}

Para limpiar y preparar la base de datos de ocupación, ha sido necesario
realizar los siguientes pasos:

\begin{enumerate}
\def\labelenumi{\arabic{enumi}.}
\item
  Eliminar la columna ``Sexo''.
\item
  Eliminar números y carácteres especiales de los nombres de las
  comunidades. Además de filtrar y eliminar el valor ``Total nacional''
  de la variable CCAA, por ser redundante respecto al resto de
  comunidades y ciudades autónomas. Posteriormente, esta variable se
  convierte a tipo factor.
\item
  Filtrar los registros correspondientes al sector ``Servicios'' dentro
  de la columna Sector.económico. Dado que la tabla se centrará
  exclusivamente en este sector, la variable se elimina por resultar
  redundante.
\item
  Convertir la variable ``Periodo'' del formato YYYYTQ a un formato de
  fecha trimestral mediante la función \texttt{as.yearqtr()}.
\item
  Normalizar la variable ``Total'', eliminando la palabra ``miles'' y
  los puntos de millares, sustituyendo la coma decimal por punto,
  convirtiendo los valores a numéricos y multiplicándolos por 1.000 para
  expresarlos en personas. De este modo, se mantiene la coherencia de
  escala con la base de datos de turismo.
\end{enumerate}

\begin{Shaded}
\begin{Highlighting}[]
\NormalTok{df\_Ocupacion\_clean }\OtherTok{\textless{}{-}}\NormalTok{  df\_Ocupacion }\SpecialCharTok{\%\textgreater{}\%} 
  \CommentTok{\#1}
  \FunctionTok{filter}\NormalTok{(Sexo }\SpecialCharTok{==} \StringTok{"Ambos sexos"}\NormalTok{) }\SpecialCharTok{\%\textgreater{}\%} 
  \FunctionTok{select}\NormalTok{(}\SpecialCharTok{{-}}\NormalTok{Sexo) }\SpecialCharTok{\%\textgreater{}\%}
  \CommentTok{\#2}
  \FunctionTok{rename}\NormalTok{(}\AttributeTok{CCAA=}\NormalTok{Comunidades.y.Ciudades.Autónomas) }\SpecialCharTok{\%\textgreater{}\%} 
  \FunctionTok{mutate}\NormalTok{(}\AttributeTok{CCAA =} \FunctionTok{str\_replace}\NormalTok{(CCAA, }\StringTok{"\^{}[0{-}9]+}\SpecialCharTok{\textbackslash{}\textbackslash{}}\StringTok{s+"}\NormalTok{, }\StringTok{""}\NormalTok{), }
  \AttributeTok{CCAA =} \FunctionTok{str\_replace}\NormalTok{(CCAA, }
                     \StringTok{"\^{}}\SpecialCharTok{\textbackslash{}\textbackslash{}}\StringTok{s*([\^{},]+?)}\SpecialCharTok{\textbackslash{}\textbackslash{}}\StringTok{s*,}\SpecialCharTok{\textbackslash{}\textbackslash{}}\StringTok{s*([\^{},]+?)}\SpecialCharTok{\textbackslash{}\textbackslash{}}\StringTok{s*$"}\NormalTok{, }\StringTok{"}\SpecialCharTok{\textbackslash{}\textbackslash{}}\StringTok{2 }\SpecialCharTok{\textbackslash{}\textbackslash{}}\StringTok{1"}\NormalTok{)) }\SpecialCharTok{\%\textgreater{}\%}
  \FunctionTok{filter}\NormalTok{(CCAA }\SpecialCharTok{!=} \StringTok{"Total Nacional"}\NormalTok{) }\SpecialCharTok{\%\textgreater{}\%}
  \FunctionTok{mutate}\NormalTok{(}\AttributeTok{CCAA=}\FunctionTok{as.factor}\NormalTok{(CCAA)) }\SpecialCharTok{\%\textgreater{}\%} 
  \CommentTok{\#3}
  \FunctionTok{filter}\NormalTok{(Sector.económico}\SpecialCharTok{==}\StringTok{\textquotesingle{}Servicios\textquotesingle{}}\NormalTok{) }\SpecialCharTok{\%\textgreater{}\%} 
  \FunctionTok{select}\NormalTok{(}\SpecialCharTok{{-}}\NormalTok{Sector.económico) }\SpecialCharTok{\%\textgreater{}\%} 
  \CommentTok{\#4 }
  \FunctionTok{mutate}\NormalTok{(}\AttributeTok{Periodo =} \FunctionTok{str\_replace}\NormalTok{(Periodo, }\StringTok{"T"}\NormalTok{, }\StringTok{"{-}"}\NormalTok{), }
    \AttributeTok{Periodo =} \FunctionTok{as.yearqtr}\NormalTok{(Periodo)) }\SpecialCharTok{\%\textgreater{}\%} 
  \CommentTok{\#5}
  \FunctionTok{mutate}\NormalTok{(}
    \AttributeTok{Total=}\FunctionTok{str\_replace\_all}\NormalTok{(Total, }\StringTok{\textquotesingle{}}\SpecialCharTok{\textbackslash{}\textbackslash{}}\StringTok{.\textquotesingle{}}\NormalTok{,}\StringTok{""}\NormalTok{),}
    \AttributeTok{Total =} \FunctionTok{str\_replace}\NormalTok{(Total, }\StringTok{","}\NormalTok{, }\StringTok{"."}\NormalTok{),}
    \AttributeTok{Total =} \FunctionTok{as.numeric}\NormalTok{(Total)}\SpecialCharTok{*}\DecValTok{1000} 
\NormalTok{  ) }\SpecialCharTok{\%\textgreater{}\%} 
  \FunctionTok{arrange}\NormalTok{(CCAA, Periodo)}
\end{Highlighting}
\end{Shaded}

\subsubsection{Análisis}\label{anuxe1lisis-1}

A continuación, se representa la \textbf{evolución trimestral del empleo
nacional en el sector servicios}. Para ello, se agrupan los datos por
periodo y se calcula el total de personas ocupadas, generando una serie
temporal mediante un gráfico de líneas.

\begin{Shaded}
\begin{Highlighting}[]
\CommentTok{\# Total nacional por trimestre}
\NormalTok{ocup\_nacional }\OtherTok{\textless{}{-}}\NormalTok{ df\_Ocupacion\_clean }\SpecialCharTok{\%\textgreater{}\%}
  \FunctionTok{group\_by}\NormalTok{(Periodo) }\SpecialCharTok{\%\textgreater{}\%}
  \FunctionTok{summarise}\NormalTok{(}\AttributeTok{Total =} \FunctionTok{sum}\NormalTok{(Total, }\AttributeTok{na.rm =} \ConstantTok{TRUE}\NormalTok{), }\AttributeTok{.groups =} \StringTok{"drop"}\NormalTok{)}

\FunctionTok{ggplot}\NormalTok{(ocup\_nacional, }\FunctionTok{aes}\NormalTok{(Periodo, Total)) }\SpecialCharTok{+}
  \FunctionTok{geom\_line}\NormalTok{(}\AttributeTok{linewidth =} \FloatTok{0.8}\NormalTok{) }\SpecialCharTok{+}
  \FunctionTok{scale\_y\_continuous}\NormalTok{(}\AttributeTok{labels =} \FunctionTok{label\_comma}\NormalTok{(}\AttributeTok{big.mark =} \StringTok{"."}\NormalTok{, }\AttributeTok{decimal.mark =} \StringTok{","}\NormalTok{))}\SpecialCharTok{+}
  \FunctionTok{labs}\NormalTok{(}\AttributeTok{title =} 
         \StringTok{"Ocupación total nacional en sector servicios"}\NormalTok{,}
       \AttributeTok{x =} \StringTok{"Periodo"}\NormalTok{, }\AttributeTok{y =} \StringTok{"Personas ocupadas"}\NormalTok{) }\SpecialCharTok{+}
  \FunctionTok{theme\_minimal}\NormalTok{()}
\end{Highlighting}
\end{Shaded}

\begin{verbatim}
## Warning: The `trans` argument of `continuous_scale()` is deprecated as of ggplot2 3.5.0.
## i Please use the `transform` argument instead.
## This warning is displayed once every 8 hours.
## Call `lifecycle::last_lifecycle_warnings()` to see where this warning was
## generated.
\end{verbatim}

\includegraphics[width=1\linewidth,height=0.3\textheight]{Informe_Grupo_F_files/figure-latex/unnamed-chunk-29-1}

Se observa una caída entre 2010 y 2014, posiblemente asociada a los
efectos de la crisis de 2008, seguida de una recuperación sostenida
hasta 2019.

En 2020 se produce un descenso notable debido al impacto de la pandemia,
tras lo cual la ocupación retoma una tendencia creciente, alcanzando
niveles superiores a los anteriores.

Además, se aprecia una estacionalidad clara, con aumentos recurrentes en
determinados trimestres, probablemente vinculados a la temporada
turística.

Además, también se analiza la \textbf{evolución trimestral del empleo en
el sector servicios por Comunidad Autónoma}, agrupando los datos por
región y periodo. El gráfico permite comparar la dinámica laboral entre
territorios a lo largo del tiempo.

\begin{Shaded}
\begin{Highlighting}[]
\NormalTok{ocup\_ccaa }\OtherTok{\textless{}{-}}\NormalTok{ df\_Ocupacion\_clean }\SpecialCharTok{\%\textgreater{}\%}
  \FunctionTok{group\_by}\NormalTok{(CCAA, Periodo) }\SpecialCharTok{\%\textgreater{}\%}
  \FunctionTok{summarise}\NormalTok{(}\AttributeTok{Total =} \FunctionTok{sum}\NormalTok{(Total, }\AttributeTok{na.rm =} \ConstantTok{TRUE}\NormalTok{), }\AttributeTok{.groups =} \StringTok{"drop"}\NormalTok{)}

\FunctionTok{ggplot}\NormalTok{(ocup\_ccaa, }\FunctionTok{aes}\NormalTok{(Periodo, Total)) }\SpecialCharTok{+}
  \FunctionTok{geom\_line}\NormalTok{(}\AttributeTok{linewidth =} \FloatTok{0.5}\NormalTok{) }\SpecialCharTok{+}
  \FunctionTok{facet\_wrap}\NormalTok{(}\SpecialCharTok{\textasciitilde{}}\NormalTok{ CCAA, }\AttributeTok{scales =} \StringTok{"free\_y"}\NormalTok{, }\AttributeTok{ncol =} \DecValTok{3}\NormalTok{) }\SpecialCharTok{+}
  \FunctionTok{scale\_y\_continuous}\NormalTok{(}\AttributeTok{labels =} \FunctionTok{label\_comma}\NormalTok{(}\AttributeTok{big.mark =} \StringTok{"."}\NormalTok{, }\AttributeTok{decimal.mark =} \StringTok{","}\NormalTok{))}\SpecialCharTok{+}
  \FunctionTok{labs}\NormalTok{(}\AttributeTok{title =} 
         \StringTok{"Ocupación en el sector servicios por Comunidad Autónoma (trimestral)"}\NormalTok{,}
       \AttributeTok{x =} \StringTok{"Periodo"}\NormalTok{, }\AttributeTok{y =} \StringTok{"Personas ocupadas"}\NormalTok{) }\SpecialCharTok{+}
  \FunctionTok{theme\_minimal}\NormalTok{() }\SpecialCharTok{+}
  \FunctionTok{theme}\NormalTok{(}\AttributeTok{panel.spacing =} \FunctionTok{unit}\NormalTok{(}\DecValTok{1}\NormalTok{, }\StringTok{"lines"}\NormalTok{),}
        \AttributeTok{strip.text =} \FunctionTok{element\_text}\NormalTok{(}\AttributeTok{size =} \DecValTok{9}\NormalTok{))}
\end{Highlighting}
\end{Shaded}

\includegraphics[width=1\linewidth,height=0.3\textheight]{Informe_Grupo_F_files/figure-latex/unnamed-chunk-30-1}

Se observa una tendencia al alza más pronunciada en comunidades como
Andalucía, Canarias, Madrid y Cataluña, principales motores del sector
servicios. Destaca una fuerte estacionalidad en Illes Balears, asociada
a la actividad turística, mientras que en la mayoría de regiones la
evolución es más estable y con un crecimiento moderado tras la pandemia.

\subsubsection{Detección de outliers}\label{detecciuxf3n-de-outliers-1}

Aplicando los tres métodos de detección de valores atípicos, al igual
que en la base de datos del turismo, se obtiene el siguiente número de
outliers identificados por cada método:

\begin{verbatim}
## # A tibble: 19 x 4
##    CCAA                       n_hampel n_boxplot n_sigma3
##    <fct>                         <int>     <int>    <int>
##  1 Extremadura                       6         8        0
##  2 Principado de Asturias            5         7        0
##  3 Castilla - La Mancha              3         0        0
##  4 La Rioja                          3         1        0
##  5 Comunidad de Madrid               2         1        0
##  6 Andalucía                         1         3        0
##  7 Cataluña                          1         0        0
##  8 Galicia                           1         0        0
##  9 País Vasco                        1         3        0
## 10 Aragón                            0         0        0
## 11 Canarias                          0         0        0
## 12 Cantabria                         0         3        0
## 13 Castilla y León                   0         0        0
## 14 Ceuta                             0         0        0
## 15 Comunidad Foral de Navarra        0         0        0
## 16 Comunitat Valenciana              0         5        0
## 17 Illes Balears                     0         0        0
## 18 Melilla                           0         0        0
## 19 Región de Murcia                  0         0        0
\end{verbatim}

El número de valores atípicos detectados en la serie de ocupación es
reducido y no presenta un patrón sistemático entre comunidades
autónomas. En la mayoría de los casos, estas observaciones reflejan
variaciones reales del empleo, asociadas al crecimiento estructural del
mercado laboral o a episodios económicos concretos, como la crisis de
2008.

A diferencia de lo ocurrido en la base de datos turística ---donde los
outliers correspondientes al periodo de la pandemia de la COVID-19 se
eliminaron por representar una disrupción excepcional---, en este caso
se considera que las fluctuaciones detectadas forman parte del
comportamiento normal de la serie.

Por tanto, no se eliminarán las observaciones atípicas y se mantendrán
en el análisis, tomando como referencia el método de detección 3-sigma,
que ofrece una estimación más estable y coherente con la naturaleza de
estos datos.

\section{ANÁLISIS BIVARIANTE}\label{anuxe1lisis-bivariante}

\#\#TODO

\textbf{Base de datos actualizada con los valores imputados}

\begin{Shaded}
\begin{Highlighting}[]
\NormalTok{AnalisisBox }\OtherTok{\textless{}{-}} \FunctionTok{readRDS}\NormalTok{(}\StringTok{"../AnalisisBox.rds"}\NormalTok{)}

\NormalTok{df }\OtherTok{\textless{}{-}} \FunctionTok{data.frame}\NormalTok{(}\FunctionTok{matrix}\NormalTok{(}\AttributeTok{nrow =} \DecValTok{0}\NormalTok{, }\AttributeTok{ncol =} \DecValTok{12}\NormalTok{))}
\NormalTok{ccaa }\OtherTok{\textless{}{-}} \FunctionTok{levels}\NormalTok{(df\_Turismo}\SpecialCharTok{$}\NormalTok{CCAA)}
\NormalTok{j }\OtherTok{\textless{}{-}} \DecValTok{1}

\ControlFlowTok{for}\NormalTok{ (x }\ControlFlowTok{in}\NormalTok{ AnalisisBox)\{}
  \ControlFlowTok{if}\NormalTok{ (}\FunctionTok{inherits}\NormalTok{(x, }\StringTok{"forecast"}\NormalTok{))\{}
\NormalTok{    df[j, ] }\OtherTok{\textless{}{-}}\NormalTok{ x}\SpecialCharTok{$}\NormalTok{mean }
    \FunctionTok{rownames}\NormalTok{(df)[j] }\OtherTok{\textless{}{-}}\NormalTok{ ccaa[[j]]}
\NormalTok{    j }\OtherTok{\textless{}{-}}\NormalTok{ j }\SpecialCharTok{+} \DecValTok{1}
\NormalTok{  \}}
\NormalTok{\}}
\end{Highlighting}
\end{Shaded}

\begin{Shaded}
\begin{Highlighting}[]
\CommentTok{\# Copia del data frame original}
\NormalTok{df\_Turismo\_corregido }\OtherTok{\textless{}{-}}\NormalTok{ df\_Turismo}

\CommentTok{\# Ordenar por CCAA y Periodo para tener los meses en orden dentro de cada comunidad}
\NormalTok{df\_Turismo\_corregido }\OtherTok{\textless{}{-}}\NormalTok{ df\_Turismo\_corregido }\SpecialCharTok{\%\textgreater{}\%}
  \FunctionTok{arrange}\NormalTok{(CCAA, Periodo)}

\CommentTok{\# Definir rango COVID (12 meses: 2020{-}03 a 2021{-}02)}
\NormalTok{inicio\_covid }\OtherTok{\textless{}{-}} \FunctionTok{as.Date}\NormalTok{(}\StringTok{"2020{-}03{-}01"}\NormalTok{)}
\NormalTok{fin\_covid    }\OtherTok{\textless{}{-}} \FunctionTok{as.Date}\NormalTok{(}\StringTok{"2021{-}02{-}01"}\NormalTok{)}

\CommentTok{\# Aseguramos que df es data.frame y tiene dimensiones}


\CommentTok{\# Número de meses que tenemos predichos (columnas de df)}
\NormalTok{n\_meses }\OtherTok{\textless{}{-}} \FunctionTok{ncol}\NormalTok{(df)}

\CommentTok{\# Comunidades en el mismo orden para df}
\NormalTok{ccaa\_niveles }\OtherTok{\textless{}{-}} \FunctionTok{levels}\NormalTok{(df\_Turismo\_corregido}\SpecialCharTok{$}\NormalTok{CCAA)}

\CommentTok{\# Opcional pero útil: asignar nombres de fila a df según las CCAA}
\ControlFlowTok{if}\NormalTok{ (}\FunctionTok{nrow}\NormalTok{(df) }\SpecialCharTok{==} \FunctionTok{length}\NormalTok{(ccaa\_niveles)) \{}
  \FunctionTok{rownames}\NormalTok{(df) }\OtherTok{\textless{}{-}}\NormalTok{ ccaa\_niveles}
\NormalTok{\}}

\CommentTok{\# Bucle: para cada CCAA, sustituir los 12 meses de COVID por los del df}
\ControlFlowTok{for}\NormalTok{ (ccaa\_actual }\ControlFlowTok{in}\NormalTok{ ccaa\_niveles) \{}
  
  \CommentTok{\# índices de las filas de esa CCAA en el periodo COVID}
\NormalTok{  idx }\OtherTok{\textless{}{-}} \FunctionTok{which}\NormalTok{(}
\NormalTok{    df\_Turismo\_corregido}\SpecialCharTok{$}\NormalTok{CCAA   }\SpecialCharTok{==}\NormalTok{ ccaa\_actual }\SpecialCharTok{\&}
\NormalTok{    df\_Turismo\_corregido}\SpecialCharTok{$}\NormalTok{Periodo }\SpecialCharTok{\textgreater{}=}\NormalTok{ inicio\_covid }\SpecialCharTok{\&}
\NormalTok{    df\_Turismo\_corregido}\SpecialCharTok{$}\NormalTok{Periodo }\SpecialCharTok{\textless{}=}\NormalTok{ fin\_covid}
\NormalTok{  )}
  
  \CommentTok{\# si hay tantas filas como meses predichos, sustituimos}
  \ControlFlowTok{if}\NormalTok{ (}\FunctionTok{length}\NormalTok{(idx) }\SpecialCharTok{==}\NormalTok{ n\_meses }\SpecialCharTok{\&\&}\NormalTok{ ccaa\_actual }\SpecialCharTok{\%in\%} \FunctionTok{rownames}\NormalTok{(df)) \{}
\NormalTok{    df\_Turismo\_corregido}\SpecialCharTok{$}\NormalTok{Total[idx] }\OtherTok{\textless{}{-}} \FunctionTok{as.numeric}\NormalTok{(df[ccaa\_actual, }\DecValTok{1}\SpecialCharTok{:}\NormalTok{n\_meses])}
\NormalTok{  \}}
\NormalTok{\}}
\end{Highlighting}
\end{Shaded}

En esta sección se abordará el análisis bivariante, con el objetivo de
relacionar las bases de datos de turismo y ocupación para estudiar
posibles vínculos entre ambas variables.

Antes de combinar los conjuntos de datos, será necesario estandarizar
ciertas variables con el fin de garantizar una fusión correcta (merge) y
obtener un marco de datos coherente para el análisis conjunto.

Así pues, se convierte la variable Periodo al formato año-trimestre,
luego se agrupan los datos por comunidad y periodo, y se calcula el
total de turistas por trimestre y CCAA. Finalmente, se ordenan los
registros por comunidad y periodo para facilitar el merge posterior.

\begin{Shaded}
\begin{Highlighting}[]
\NormalTok{turismo\_to\_merge }\OtherTok{\textless{}{-}}\NormalTok{ df\_Turismo\_corregido }\SpecialCharTok{\%\textgreater{}\%}
  \FunctionTok{mutate}\NormalTok{(}\AttributeTok{Periodo =} \FunctionTok{as.yearqtr}\NormalTok{(Periodo)) }\SpecialCharTok{\%\textgreater{}\%}
  \FunctionTok{group\_by}\NormalTok{(CCAA, Periodo) }\SpecialCharTok{\%\textgreater{}\%}
  \FunctionTok{summarise}\NormalTok{(}\AttributeTok{Total\_turismo =} \FunctionTok{sum}\NormalTok{(Total, }\AttributeTok{na.rm =} \ConstantTok{TRUE}\NormalTok{), }\AttributeTok{.groups =} \StringTok{"drop"}\NormalTok{) }\SpecialCharTok{\%\textgreater{}\%} 
  \FunctionTok{arrange}\NormalTok{(CCAA, Periodo)}
\end{Highlighting}
\end{Shaded}

Mediante la función \texttt{merge()}, se unen ambos conjuntos de datos
utilizando como claves comunes las variables CCAA y Periodo.

Antes de la unión, en la base de ocupación se renombra la variable Total
como Numero\_ocupados para distinguirla del total de turistas y evitar
ambigüedades en el nuevo dataset combinado (df\_Turismo\_Ocupacion).

\begin{Shaded}
\begin{Highlighting}[]
\NormalTok{df\_Turismo\_Ocupacion }\OtherTok{\textless{}{-}} \FunctionTok{merge}\NormalTok{(}
  \AttributeTok{x =}\NormalTok{ turismo\_to\_merge,}
  \AttributeTok{y =}\NormalTok{ (df\_Ocupacion\_clean }\SpecialCharTok{\%\textgreater{}\%} 
    \FunctionTok{rename}\NormalTok{(}\AttributeTok{Numero\_ocupados =}\NormalTok{ Total)),}
  \AttributeTok{by.x =} \FunctionTok{c}\NormalTok{(}\StringTok{"CCAA"}\NormalTok{, }\StringTok{"Periodo"}\NormalTok{),}
  \AttributeTok{by.y =} \FunctionTok{c}\NormalTok{(}\StringTok{"CCAA"}\NormalTok{, }\StringTok{"Periodo"}\NormalTok{))}
\end{Highlighting}
\end{Shaded}

Finalmente, una vez preparadas y combinadas las bases de datos, se
guardan los objetos resultantes en formato \texttt{.RData}.

Esto permite conservar los datos limpios y listos para su análisis,
facilitando su carga posterior sin necesidad de repetir el proceso de
limpieza y fusión.

\begin{center}\rule{0.5\linewidth}{0.5pt}\end{center}

\textbf{Correlaciones de Pearson}

Como primer paso, se calcula la correlación de Pearson entre el número
de turistas y el empleo en el sector servicios para cada Comunidad
Autónoma.

El objetivo es medir la intensidad y dirección de la relación lineal
entre ambas variables, obteniendo tanto el coeficiente de correlación
como su p-valor para evaluar la significatividad estadística.

Además, se incorpora un indicador de nivel de significación (p
\textless{} 0.05), que permite identificar en qué comunidades la
relación es estadísticamente relevante.

Los resultados se ordenan de mayor a menor correlación, destacando
aquellas regiones donde el vínculo entre turismo y empleo es más fuerte
y significativo.

\begin{verbatim}
## # A tibble: 19 x 4
##    CCAA                          cor  p_value sign 
##    <fct>                       <dbl>    <dbl> <chr>
##  1 Illes Balears              0.706  6.24e-12 Sí   
##  2 Canarias                   0.624  6.30e- 9 Sí   
##  3 Comunidad de Madrid        0.526  2.45e- 6 Sí   
##  4 Aragón                     0.484  1.86e- 5 Sí   
##  5 País Vasco                 0.482  2.06e- 5 Sí   
##  6 Castilla y León            0.473  3.17e- 5 Sí   
##  7 Castilla - La Mancha       0.458  5.99e- 5 Sí   
##  8 Extremadura                0.416  3.05e- 4 Sí   
##  9 Galicia                    0.407  4.29e- 4 Sí   
## 10 Cantabria                  0.402  5.07e- 4 Sí   
## 11 Comunidad Foral de Navarra 0.380  1.09e- 3 Sí   
## 12 Andalucía                  0.374  1.30e- 3 Sí   
## 13 Región de Murcia           0.350  2.80e- 3 Sí   
## 14 Comunitat Valenciana       0.278  1.88e- 2 Sí   
## 15 Cataluña                   0.247  3.76e- 2 Sí   
## 16 La Rioja                   0.213  7.44e- 2 No   
## 17 Principado de Asturias     0.213  7.45e- 2 No   
## 18 Melilla                    0.105  3.82e- 1 No   
## 19 Ceuta                      0.0834 4.89e- 1 No
\end{verbatim}

\includegraphics[width=1\linewidth,height=0.3\textheight]{Informe_Grupo_F_files/figure-latex/unnamed-chunk-40-1}

Se observa que Illes Balears presenta la mayor correlación positiva (r =
0.68) entre turismo y empleo en el sector servicios, reflejando su
fuerte dependencia del turismo estival. En el extremo opuesto, Melilla
muestra una correlación negativa (r = -0.24), probablemente por una
menor relación directa entre la llegada de turistas y el empleo local.

En general, las correlaciones son positivas pero de magnitud moderada,
indicando que ambas variables tienden a evolucionar conjuntamente,
aunque con distinta intensidad según la región.

En la mayoría de comunidades, la relación es estadísticamente
significativa (p \textless{} 0.05), mientras que en Ceuta, La Rioja,
Cataluña y Asturias no se observa significatividad, posiblemente por el
menor peso del turismo en su economía o por una mayor volatilidad en los
datos.

Por último, cabe señalar que la correlación no implica causalidad: los
resultados reflejan asociación, pero no necesariamente una relación
directa de causa y efecto.

\textbf{Análisis de regresión lineal simple}

Además, modelamos para cada comunidad una regresión simple con el
objetivo de conocer la \textbf{bondad de ajuste} (\texttt{R\^{}2}), lo
cual permitea valorar qué proporción de la variabilidad en el empleo es
explicada por el turismo.

\begin{verbatim}
## # A tibble: 19 x 2
## # Rowwise: 
##    CCAA                            R2
##    <fct>                        <dbl>
##  1 Andalucía                  0.140  
##  2 Aragón                     0.235  
##  3 Canarias                   0.389  
##  4 Cantabria                  0.162  
##  5 Castilla - La Mancha       0.209  
##  6 Castilla y León            0.223  
##  7 Cataluña                   0.0612 
##  8 Ceuta                      0.00695
##  9 Comunidad de Madrid        0.277  
## 10 Comunidad Foral de Navarra 0.144  
## 11 Comunitat Valenciana       0.0774 
## 12 Extremadura                0.173  
## 13 Galicia                    0.166  
## 14 Illes Balears              0.498  
## 15 La Rioja                   0.0454 
## 16 Melilla                    0.0111 
## 17 País Vasco                 0.233  
## 18 Principado de Asturias     0.0454 
## 19 Región de Murcia           0.122
\end{verbatim}

\includegraphics[width=1\linewidth,height=0.3\textheight]{Informe_Grupo_F_files/figure-latex/unnamed-chunk-42-1}

En general, se observa que los valores son moderados o bajos, lo que
indica que, aunque existe cierta relación entre ambas variables, el
turismo no explica completamente la evolución del empleo, y
probablemente influyen otros factores económicos o estructurales.

Las comunidades con mayor ajuste del modelo son Aragón (\(R^2 = 0.25\)),
Castilla y León (\(R^2 = 0.24\)), Cantabria (\(R^2 = 0.22\)) y Canarias
(\(R^2 = 0.22\)), donde alrededor de una cuarta parte de la variabilidad
en el empleo se asocia con los cambios en el turismo.

Por el contrario, en Ceuta (\(R^2 \approx 0\)) y Cataluña
(\(R approx 0.05\)) el modelo lineal apenas explica la variación en el
empleo, lo que sugiere una relación débil o inexistente entre ambas
variables en estas regiones.

A partir del modelo de regresión lineal calculado anteriormente, se
estima la \textbf{elasticidad} del empleo respecto al turismo,
utilizando la pendiente de la recta de regresión como medida del impacto
del turismo sobre el empleo en cada comunidad autónoma.

\begin{verbatim}
## # A tibble: 19 x 3
##    CCAA                       elasticidad int       
##    <fct>                            <dbl> <chr>     
##  1 Canarias                        0.396  Inelástico
##  2 Castilla - La Mancha            0.211  Inelástico
##  3 Comunidad de Madrid             0.207  Inelástico
##  4 Extremadura                     0.156  Inelástico
##  5 Región de Murcia                0.130  Inelástico
##  6 Illes Balears                   0.117  Inelástico
##  7 Aragón                          0.109  Inelástico
##  8 Andalucía                       0.0949 Inelástico
##  9 Comunitat Valenciana            0.0825 Inelástico
## 10 País Vasco                      0.0816 Inelástico
## 11 Comunidad Foral de Navarra      0.0777 Inelástico
## 12 Castilla y León                 0.0752 Inelástico
## 13 Melilla                         0.0563 Inelástico
## 14 Cataluña                        0.0528 Inelástico
## 15 La Rioja                        0.0452 Inelástico
## 16 Cantabria                       0.0418 Inelástico
## 17 Galicia                         0.0407 Inelástico
## 18 Ceuta                           0.0295 Inelástico
## 19 Principado de Asturias          0.0189 Inelástico
\end{verbatim}

Los valores de elasticidad, comprendidos entre -0.09 y 0.19, muestran
que en todas las comunidades autónomas la relación entre turismo y
empleo es inelástica (ya que \textless1), es decir, el empleo en el
sector servicios apenas varía ante los cambios en la actividad
turística. Esto sugiere que un aumento del turismo no se traduce
proporcionalmente en un mayor nivel de empleo.

Esta baja respuesta puede deberse a que parte del empleo es estructural
o fijo, a la existencia de desfases temporales entre el incremento del
turismo y la contratación de personal, o a que las variables se miden en
frecuencias distintas (mensual para el turismo y trimestral para el
empleo), lo que suaviza la relación observada.

En regiones como Canarias (\(\approx 0.19\)), donde la elasticidad es
algo mayor, el empleo muestra una mayor sensibilidad a las variaciones
del turismo, coherente con su elevada dependencia económica de esta
actividad.

Aunque las correlaciones y la bondad de ajuste indicaban una relación
fuerte y significativa entre turismo y empleo, las elasticidades
calculadas son bajas. Esto no representa una contradicción, sino una
diferencia en la interpretación: mientras la correlación mide la fuerza
de la relación, la elasticidad cuantifica su intensidad relativa.

En otras palabras, aunque el turismo y el empleo evolucionan de forma
coordinada, el impacto porcentual del turismo sobre el empleo es
reducido, lo que sugiere que otros factores ---como la estacionalidad o
la estabilidad estructural del mercado laboral--- también influyen en la
evolución del empleo.

\section{CONCLUSIONES}\label{conclusiones}

%%%%%%%%%%%%%%%%%%%%%%%%%%%%%%%%%%%%%%%%%%

\vspace{6pt}

%%%%%%%%%%%%%%%%%%%%%%%%%%%%%%%%%%%%%%%%%%
%% optional

% Only for the journal Methods and Protocols:
% If you wish to submit a video article, please do so with any other supplementary material.
% \supplementary{The following supporting information can be downloaded at: \linksupplementary{s1}, Figure S1: title; Table S1: title; Video S1: title. A supporting video article is available at doi: link.}

% Only for journal Hardware:
% If you wish to submit a video article, please do so with any other supplementary material.
% \supplementary{The following supporting information can be downloaded at: \linksupplementary{s1}, Figure S1: title; Table S1: title; Video S1: title.\vspace{6pt}\\
%\begin{tabularx}{\textwidth}{lll}
%\toprule
%\textbf{Name} & \textbf{Type} & \textbf{Description} \\
%\midrule
%S1 & Python script (.py) & Script of python source code used in XX \\
%S2 & Text (.txt) & Script of modelling code used to make Figure X \\
%S3 & Text (.txt) & Raw data from experiment X \\
%S4 & Video (.mp4) & Video demonstrating the hardware in use \\
%... & ... & ... \\
%\bottomrule
%\end{tabularx}
%}

%%%%%%%%%%%%%%%%%%%%%%%%%%%%%%%%%%%%%%%%%%





% Only for journal Nursing Reports
%\publicinvolvement{Please describe how the public (patients, consumers, carers) were involved in the research. Consider reporting against the GRIPP2 (Guidance for Reporting Involvement of Patients and the Public) checklist. If the public were not involved in any aspect of the research add: ``No public involvement in any aspect of this research''.}

% Only for journal Nursing Reports
%\guidelinesstandards{Please add a statement indicating which reporting guideline was used when drafting the report. For example, ``This manuscript was drafted against the XXX (the full name of reporting guidelines and citation) for XXX (type of research) research''. A complete list of reporting guidelines can be accessed via the equator network: \url{https://www.equator-network.org/}.}

% Only for journal Nursing Reports
%\guidelinesstandards{Please add a statement indicating which reporting guideline was used when drafting the report. For example, ``This manuscript was drafted against the XXX (the full name of reporting guidelines and citation) for XXX (type of research) research''. A complete list of reporting guidelines can be accessed via the equator network: \url{https://www.equator-network.org/}.}



%%%%%%%%%%%%%%%%%%%%%%%%%%%%%%%%%%%%%%%%%%
%% Optional

%% Only for journal Encyclopedia
%\entrylink{The Link to this entry published on the encyclopedia platform.}


%%%%%%%%%%%%%%%%%%%%%%%%%%%%%%%%%%%%%%%%%%
%% Optional
%%%%%%%%%%%%%%%%%%%%%%%%%%%%%%%%%%%%%%%%%%
\begin{adjustwidth}{-\extralength}{0cm}

%\printendnotes[custom] % Un-comment to print a list of endnotes



% If authors have biography, please use the format below
%\section*{Short Biography of Authors}
%\bio
%{\raisebox{-0.35cm}{\includegraphics[width=3.5cm,height=5.3cm,clip,keepaspectratio]{Definitions/author1.pdf}}}
%{\textbf{Firstname Lastname} Biography of first author}
%
%\bio
%{\raisebox{-0.35cm}{\includegraphics[width=3.5cm,height=5.3cm,clip,keepaspectratio]{Definitions/author2.jpg}}}
%{\textbf{Firstname Lastname} Biography of second author}

%%%%%%%%%%%%%%%%%%%%%%%%%%%%%%%%%%%%%%%%%%
%% for journal Sci
%\reviewreports{\\
%Reviewer 1 comments and authors’ response\\
%Reviewer 2 comments and authors’ response\\
%Reviewer 3 comments and authors’ response
%}
%%%%%%%%%%%%%%%%%%%%%%%%%%%%%%%%%%%%%%%%%%
\PublishersNote{}
\end{adjustwidth}


\end{document}
